\chapter{Anwendungen}
\label{chapter-anwendungen}
Asymmetrisches Kryptosystem“ ist ein Oberbegriff für Public-Key-Verschlüsselungsverfahren, Public-Key-Authentifizierung und digitale Signaturen. Diese Verfahren werden heutzutage z. B. im E-Mail-Verkehr (OpenPGP, S/MIME) ebenso wie in kryptografischen Protokollen wie SSH oder SSL/TLS verwendet. In größerem Umfang eingesetzt wird beispielsweise das Protokoll https zur sicheren Kommunikation eines Web-Browsers mit einem Server.
Zur Verschlüsselung wird der öffentliche Schlüssel auf den zu verschlüsselnden Text angewandt. Der verschlüsselte Text wird dann vom Schlüsselinhaber mit dem privaten Schlüssel wieder entschlüsselt.
Digitale Signaturen werden u. a. zur sicheren Abwicklung von Geschäften im Internet eingesetzt. Hier ermöglichen sie die Prüfung der Identität der Vertragspartner und der Unverfälschtheit der ausgetauschten Daten (Elektronische Signatur). Dazu ist meist noch eine Public-Key-Infrastruktur notwendig, die die Gültigkeit der verwendeten Schlüssel durch Zertifikate bestätigt.
Zum Erstellen einer Signatur wird ein Hashwert aus der zu verschickenden Nachricht gebildet und mit dem privaten Schlüssel signiert. Nachricht und Signatur werden dann zum Empfänger geschickt, wobei die eigentliche Nachricht nicht verschlüsselt zu sein braucht, da es sich hierbei um eine Signatur (Schaffen von Integrität und Authentizität) und nicht um Verschlüsselung (Schaffen von Vertraulichkeit) handelt.
Zum Verifizieren der Signatur wird die empfangene Signatur des Hashwertes mit dem öffentlichen Schlüssel geprüft. Ist die Verifizierung erfolgreich, so kann davon ausgegangen werden, dass die Nachricht vom Besitzer des privaten Schlüssels stammt und dass die Nachricht bei der Übertragung nicht manipuliert wurde.
\section{Public-Key-Verschlüssungsverfahren}
Ein Public-Key-Verschlüsselungsverfahren ist ein asymmetrisches Verschlüsselungsverfahren, also ein kryptographisches Verfahren, um mit einem öffentlichen Schlüssel einen Klartext in einen Geheimtext umzuwandeln, aus dem der Klartext mit einem geheimen Schlüssel wieder gewonnen werden kann.
Der geheime Schlüssel muss geheim gehalten werden, und es muss praktisch unmöglich sein, ihn aus dem öffentlichen Schlüssel zu berechnen. Der öffentliche Schlüssel muss jedem zugänglich sein, der eine verschlüsselte Nachricht an den Besitzer des geheimen Schlüssels senden will. Dabei muss sichergestellt sein, dass der öffentliche Schlüssel auch wirklich dem Empfänger zugeordnet ist.
\section{Public-Key-Authentifizierung}
Die Public-Key-Authentifizierung ist eine Authentifizierungsmethode, die unter anderem von SSH und OpenSSH verwendet wird, um Benutzer mit Hilfe eines Schlüsselpaars, bestehend aus privatem und öffentlichem Schlüssel, an einem Server anzumelden. Ein solches Schlüsselpaar ist wesentlich schwerer zu kompromittieren als ein Kennwort.
Bei einer Authentifizierung mittels eines Kennworts wird dieses Kennwort, oder dessen Hash-Wert, auf einem Server gespeichert. Hat jemand Zugang zur Kennwortdatei auf diesem Server, gelangt er damit im ersten Fall auch in den Besitz des Kennworts. Im zweiten Fall kann er, mit Hilfe entsprechender Software, eine Zeichenkombination finden, die den gleichen Hash-Wert wie das Kennwort ergibt. Wird das gleiche Kennwort zur Anmeldung auf mehreren Systemen benutzt, so sind damit alle diese Systeme kompromittiert.
Im Gegensatz dazu wird bei der Public-Key-Authentifizierung nur der öffentliche Schlüssel auf einem Server gespeichert. Der private Schlüssel wird auf dem eigenen Rechner gespeichert, kann damit geheim gehalten und zusätzlich mit einer Kennung verschlüsselt werden. Die Kennung kann aus mehreren Wörtern bestehen (im Englischen passphrase).
Die Berechnung des privaten Schlüssels aus dem öffentlichen ist je nach Wahl der Länge des Schlüssels sehr aufwändig bis praktisch unmöglich.
Der öffentliche Schlüssel kann auch zur automatischen Anmeldung (Authentifizierung) genutzt werden. Dabei entfällt der interaktive Dialog zur Kennworteingabe. Hierzu wird auf der User-Seite mit Hilfe seines privaten Schlüssels eine Signatur erzeugt, welche daraufhin auf der Server-Seite mit dem dort hinterlegten öffentlichen Schlüssel des Users verifiziert wird.[1] Das ermöglicht eine Anmeldung in ohne Benutzereingabe ablaufenden Skripten, und beim automatisierten Kopieren von Dateien, beispielsweise mit Secure Copy.
\section{digitale Signatur}
Eine digitale Signatur, auch digitales Signaturverfahren, ist ein asymmetrisches Kryptosystem, bei dem ein Sender mit Hilfe eines geheimen Signaturschlüssels (dem Private Key) zu einer digitalen Nachricht (d. h. zu beliebigen Daten) einen Wert berechnet, der ebenfalls digitale Signatur genannt wird. Dieser Wert ermöglicht es jedem, mit Hilfe des öffentlichen Verifikationsschlüssels (dem Public Key) die nichtabstreitbare Urheberschaft und Integrität der Nachricht zu prüfen. Um eine mit einem Signaturschlüssel erstellte Signatur einer Person zuordnen zu können, muss der zugehörige Verifikationsschlüssel dieser Person zweifelsfrei zugeordnet sein.
\section{SSH}
Secure Shell oder SSH bezeichnet sowohl ein Netzwerkprotokoll als auch entsprechende Programme, mit deren Hilfe man auf eine sichere Art und Weise eine verschlüsselte Netzwerkverbindung mit einem entfernten Gerät herstellen kann. Häufig wird diese Methode verwendet, um lokal eine entfernte Kommandozeile verfügbar zu machen, das heißt, auf einer lokalen Konsole werden die Ausgaben der entfernten Konsole ausgegeben und die lokalen Tastatureingaben werden an den entfernten Rechner gesendet. Genutzt werden kann dies beispielsweise zur Fernwartung eines in einem entfernten Rechenzentrum stehenden Servers. Die neuere Protokoll-Version SSH-2 bietet weitere Funktionen wie Datenübertragung per SFTP.