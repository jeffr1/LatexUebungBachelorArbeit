%! TEX root = thesis.tex

\chapter{Geschichte}
\label{chapter-geschichte}
Den ersten Schritt zur Entwicklung asymmetrischer Verfahren machte Ralph Merkle 1974 mit dem nach ihm benannten Merkles Puzzle, das aber erst 1978 veröffentlicht wurde. Unter dem Einfluss dieser Arbeit entwickelten Whitfield Diffie und Martin Hellman im Jahr 1976 den Diffie-Hellman-Schlüsselaustausch.\cite{ndic} Im Sommer 1975 veröffentlichten Diffie und Hellman eine Idee zur asymmetrischen Verschlüsselung, ohne jedoch ein genaues Verfahren zu kennen. Das erste asymmetrische Verschlüsselungsverfahren wurde 1977 von Ronald L. Rivest, Adi Shamir und Leonard M. Adleman entwickelt und nach ihnen RSA-Verfahren genannt. Nach heutiger Terminologie ist dieses Verfahren eine Falltürpermutation, die sowohl zur Konstruktion von Verschlüsselungsverfahren als auch von Signaturverfahren eingesetzt werden kann.
Unabhängig von den Entwicklungen in der wissenschaftlichen Kryptologie wurde Anfang der 1970er Jahre von drei Mitarbeitern des britischen Government Communications Headquarters, James H. Ellis, Clifford Cocks und Malcolm Williamson, sowohl ein dem späteren Diffie-Hellman-Schlüsselaustausch als auch ein dem RSA-Kryptosystem ähnliches asymmetrisches Verfahren entwickelt, welches aber aus Geheimhaltungsgründen nicht publiziert und auch nicht zum Patent angemeldet wurde.\ref{pa}
\begin{table} 
\caption{Pantentanmeldung}
\label{pa}
 \begin{tabular}{c|l}
  \rowcolor{gray!20}Jahr & Kryptosystem \\\hline
1977&	RSA\\
1978&	Merkle-Hellman\\
1978&	McEliece\\
1979&	Rabin\\
1984	&Chor-Rivest\\
1985&	Elgamal\\

 \end{tabular}
\end{table}