%! TEX root = thesis.tex
\chapter{Zusammenfassung}
Ein asymmetrisches Kryptosystem oder Public-Key-Kryptosystem ist ein kryptographisches Verfahren, bei dem im Gegensatz zu einem symmetrischen Kryptosystem die kommunizierenden Parteien keinen gemeinsamen geheimen Schlüssel zu kennen brauchen. Jeder Benutzer erzeugt sein eigenes Schlüsselpaar, das aus einem geheimen Teil (privater Schlüssel) und einem nicht geheimen Teil (öffentlicher Schlüssel) besteht. Der öffentliche Schlüssel ermöglicht es jedem, Daten für den Besitzer des privaten Schlüssels zu verschlüsseln, dessen digitale Signaturen zu prüfen oder ihn zu authentifizieren. Der private Schlüssel ermöglicht es seinem Besitzer, mit dem öffentlichen Schlüssel verschlüsselte Daten zu entschlüsseln, digitale Signaturen zu erzeugen oder sich zu authentisieren.
Im folgenden weitere Literatur in Form von Büchern genannt\cite{krsog}, \cite{gb}, \cite{ak}.
Der noch fehlende Bericht in einen Konferenzband \cite{DBLP:conf/pkc/BernhardFW16}.