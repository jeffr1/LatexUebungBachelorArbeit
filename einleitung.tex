%!tex root = thesis.tex

\chapter{Einleitung }
Die theoretische Grundlage für asymmetrische Kryptosysteme sind Falltürfunktionen, also Funktionen, die leicht zu berechnen, aber ohne ein Geheimnis (die „Falltür“) praktisch unmöglich zu invertieren sind. Der öffentliche Schlüssel ist dann eine Beschreibung der Funktion, der private Schlüssel ist die Falltür. Eine Voraussetzung ist natürlich, dass der private Schlüssel aus dem öffentlichen nicht berechnet werden kann. Damit das Kryptosystem verwendet werden kann, muss der öffentliche Schlüssel dem Kommunikationspartner bekannt sein.
Der entscheidende Vorteil von asymmetrischen Verfahren ist, dass sie das Schlüsselverteilungsproblem vermindern. Bei symmetrischen Verfahren muss vor der Verwendung ein Schlüssel über einen sicheren, d. h. abhörsicheren und manipulationsgeschützten Kanal ausgetauscht werden. Da der öffentliche Schlüssel nicht geheim ist, braucht bei asymmetrischen Verfahren der Kanal nicht abhörsicher zu sein; wichtig ist nur, dass der öffentliche Schlüssel dem Inhaber des dazugehörigen geheimen Schlüssels zweifelsfrei zugeordnet werden kann. Dazu kann beispielsweise eine vertrauenswürdige Zertifizierungsstelle ein digitales Zertifikat ausstellen, welches den öffentlichen Schlüssel dem privaten Schlüssel(inhaber) zuordnet. Als Alternative dazu kann auch ohne zentrale Stelle durch gegenseitiges Zertifizieren von Schlüsseln ein Vertrauensnetz (Web of Trust) aufgebaut werden.

\section{Aufbau der Arbeit}
Neben dieser Einleitung und der Zusammenfassung am Ende gliedert sich diese Arbeit in die folgenden drei Kapitel.
\begin{description}
  \item[\ref{chapter-Sicherheit}] erläutert Sicherheit 
  \item[\ref{chapter-aspeckte}] wird näher über praktische Aspekte berichtet 
  \item[\ref{chapter-geschichte}] wird näher die geschichtliche Entwicklung beleuchtet.
  \item[\ref{chapter-anwendungen}] werden Anwendungen genannt
\end{description}